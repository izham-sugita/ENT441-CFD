%%%%%%%%%%%%%%%%%%
%%% QUESTION 3 %%%
%%%%%%%%%%%%%%%%%%
%TODO Question3
%\clearpage		% page break
\question{}\label{q3}
%\partQuestion{}

The one-dimensional linear advection equation is the canonical example of transport phenomena. The equation is an excellent model equation to test the behavior of numerical method. The equation is given as below.  
	
	\translation{Persamaan adveksi linear satu dimensi merupakan satu contoh persamaan berkanun bagi fenomena pengangkutan. Persamaan ini juga merupakan satu persamaan model yang sangat cemerlang untuk menguji sifat-sifat kaedah berangka. Persamaan ini diberikan seperti dibawah.}
	
	\begin{equation}
	\frac{\partial u}{\partial t} + c \frac{\partial u}{\partial x} = 0, \quad c > 0 \nonumber
	\end{equation}		

\listbeginx	% start 1st level question			

    \item \label{q3a} The finite volume method can be applied to numerically solve the linear advection equation. Answer the question below regarding the finite volume method.	
	\translation{Kaedah isipadu terhingga boleh digunakan untuk menyelesaikan persamaan adveksi linear secara berangka. Jawab soalan-soalan dibawah berkenaan kaedah isipadu terhingga.}

\listbegin
\item \label{q3a1} Apply the finite volume method to derive the semi-discrete form of the advection equation.

\translation{Gunakan kaedah isipadu terhingga untuk menerbitkan persamaan separa-diskrit untuk persamaan adveksi tersebut.}

\qmarks{5}

\item \label{q3a2} Considering a uniform control volume, analyze the possible discretization method for the semi-discrete equation in Q\ref{q3}\ref{q3a}\ref{q3a1}.

\translation{Dengan menganggap isipadu kawalan yang sekata, analisa kaedah diskrit yang mungkin digunapakai untuk persamaan separuh diskrit Q\ref{q3}\ref{q3a}.}

\qmarks{5}

\listclose

		
	\item \label{q3b} From the discrete equation in Q\ref{q3}\ref{q3a}, analyze the numerical stability condition using complex Fourier analysis.   
	
	\translation{Daripada persamaan diskrit Q\ref{q3}\ref{q3a}, analisa syarat kestabilan berangka menggunakan nombor kompleks Fourier.}

	\qmarks{10}
		
	\item What is the conclusion from Q\ref{q3}\ref{q3b}? Is the method stable? Propose a solution if the method is not stable.   
	
	\translation{Apakah kesimpulan daripada Q\ref{q3}\ref{q3b}? Adakah kaedah ini stabil? Cadangkan satu penyelesaian jika kaedah ini tidak stabil.}

	\qmarks{5}

\listclose	% close 1st level question
