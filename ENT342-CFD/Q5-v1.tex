%%%%%%%%%%%%%%%%%%
%%% QUESTION 5 %%%
%%%%%%%%%%%%%%%%%%
%TODO Question5
% \clearpage		% page break
\question{} \label{q5}
%\partQuestion{}

	The time-dependent one-dimensional heat equation is a second-order partial differential equation. The second-order term infers that the solution propagation will be restricted to the order of $\mathcal{O}(1/\Delta h^{2})$, where $\Delta h$ is the mesh spacing. This put a severe stability restriction to explicit numerical scheme when solving this equation. Answer the questions referring to the time-dependent one-dimensional heat equation below.
	
	\translation{Persamaan haba satu-dimensi bersandar masa merupakan persamaan kebezaan separa darjah kedua. Terma darjah kedua bererti propagasi solusi disekat pada darjah $\mathcal{O}(1/\Delta h^{2})$, dimana $\Delta h$ adalah jarak antara grid. Ini menyebabkan penyekatan syarat kestabilan yang teruk bagi kaedah berangka tersurat ketika menyelesaikan persamaan ini. Jawab soalan-soalan berkaitan dengan persamaan haba satu-dimensi bersandar masa dibawah. }

\begin{equation}
\frac{\partial T}{\partial t} = \kappa \left( \frac{\partial^2 T}{\partial x^2} \right) \nonumber
\end{equation}		
		
		\listbeginx

\item \label{q5a} Construct the discretized equation for the heat equation using \textit{forward} Euler time integration and centered difference scheme for the spatial term. Analyze the discrete equation to derive the numerical stability condition.

\translation {Bina persamaan diskrit untuk persamaan haba itu menggunakan kaedah integrasi masa Euler kehadapan dan kaedah pembezaan pertengahan untuk terma ruang. Analisakan persamaan diskrit itu untuk menerbitkan syarat kestabilan berangka.}
		
\qmarks{10}

\item \label{q5b} From the discrete equation obtained in Q\ref{q5}\ref{q5a} modify  the time integration scheme to \textit{backward} Euler time integration. Analyze the discrete equation to derive the numerical stability condition. Justify the merit of using \textit{backward} Euler time integration.     

\translation{Ubahsuaikan kaedah integrasi masa persamaan diskrit yang diperolehi oleh Q\ref{q5}\ref{q5a} kepada kaedah integrasi masa Euler kemunduran. Analisakan persamaan ini untuk menerbitkan syarat kestabilan berangka. Justifikasikan merit menggunakan kaedah integrasi masa Euler kemunduran.}


\qmarks{10}

	
	\item The discrete equation from Q\ref{q5}\ref{q5b} contains more unknown at the next time step than at the previous time step. This condition necessitates the construction of simultaneous linear equations. Construct the matrix-vector form for the simultaneous linear equations.  
	
	\translation{Persamaan diskrit daripada Q\ref{q5}\ref{q5b} mengandungi lebih banyak terma di masa seterusnya daripada masa sebelumnya. Keadaan ini memerlukan pembinaan persamaan linear serentak. Bina persamaan linear serentak itu dalam bentuk matriks-vektor.}

	\qmarks{5}



\listclose	% close 1st level question
