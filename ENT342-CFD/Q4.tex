%%%%%%%%%%%%%%%%%%
%%% QUESTION 4 %%%
%%%%%%%%%%%%%%%%%%
%TODO Question4
\clearpage		% page break
\question{}
%\partQuestion{}

\listbeginx	% start 1st level question

	\item The relationship between the time rates of change of an extensive quantity for a control volume can be expressed by the Reynolds transport theorem.  
	
	\translation{Hubungan antara kadar perubahan pada masa untuk kuantiti yang meluas untuk sesuatu isipadu kawalan dapat dijelaskan melalui teorem pengankutan Reynolds.}
		
		\listbegin

\item Define the extensive quantity for the Reynolds transport theorem. Give two examples.

\translation {Definisikan kuantiti meluas untuk teorem pengangkutan Reynolds. Berikan dua contoh.}
		
\qmarks{2}

\item The mathematical expression of the Reynolds transport theorem is given as in equation below with $\rho$, $\mathcal{V}$, $\overrightarrow{V}$, $\overrightarrow{n}$ as density, volume of the control volume, velocity vector and the normal vector outfacing the control volume, respectively. Given $B_{sys}$ as the extensive quantity, define $b$ in the equation given.
  
\translation{Ekspresi matematik untuk teorem pengangkutan Reynolds diberikan oleh persamaan dibawah dimana $\rho$, $\mathcal{V}$, $\overrightarrow{V}$, $\overrightarrow{n}$ adalah ketumpatan, isipadu kepada isipadu kawalan, vektor halaju dan vektor normal yang menghadap keluar isipadu kawalan, masing-masingnya. Diberikan $B_{sys}$ sebagai kuantiti meluas, definisikan $b$ dalam persamaan tersebut.}
\begin{center}
\begin{equation}
\nonumber
\frac{dB_{sys}}{dt} = \frac{d}{dt}\int_{CV} \rho bd\mathcal{V} + \int_{CS} \rho b \overrightarrow{V} \boldsymbol{\cdot} \overrightarrow{n} dA
\end{equation}
\end{center}

\qmarks{2}

	\item Demonstrate the interpretation of force acting on a fluid from the Reynolds transport theorem.	
	
	\translation {Tunjukkan demonstrasi yang menginterpretasikan daya yang bertindak keatas bendalir daripada teorem transport Reynolds.}	

\qmarks{3}

\clearpage

	\item A water jet striking a stationary plate is a good example of momentum analysis and Reynolds transport theorem application. Refering to \textbf{Figure \ref{fig:jetStrikingPlate}}, the value of $\overrightarrow{V}_{1}$ is 30 m/s and the mass flow rate is given as 10 kg/s. The outflow $\overrightarrow{V}_{2}$ is given as 30 m/s. Determine the force needed to keep the plate stationary. 		

	\translation {Jet air menembak plat tetap merupakan satu contoh baik untuk aplikasi analisa momentum dan teorem pengangkutan Reynolds. Merujuk kepada Gambarajah \ref{fig:jetStrikingPlate}, nilai $\overrightarrow{V}_{1}$ diberikan 30 m/s dan kadar aliran jisim diberikan sebagai 10 kg/s. Aliran keluar $\overrightarrow{V}_{2}$ diberikan sebagai 30 m/s. Tentukan daya yang diperlukan untuk mengekalkan kedudukan plat itu.}		

\qmarks{3}
		
		\listclose

		\begin{figure}[H] % H means, to put figure here after the code
		\centering
		\includegraphics[width=0.5\textwidth]{jetStrikingPlate}
		\caption{\rajah}
		\label{fig:jetStrikingPlate}
	    \end{figure}	

	
	\clearpage 
%	\nextpage 

	
	\item \textbf{Figure \ref{fig:WindPowerGeneration}} shows a wind generator with a 10 m-diameter blade span which is specified to a minimum wind speed of 10 km/h at which it will generates 0.5 kW of electric power. Air density is given as 1.217 $kg/m^{3}$. Answer the questions below regarding the wind turbine. 
	
	\translation{Gambarajah \ref{fig:WindPowerGeneration} menunjukkan satu penjana angin dengan bilah berdiameter 10 m berspecifikasikan untuk minimum halaju angin 10 km/h dimana ia akan menjanakan elektrik sebanyak 0.5 kW. Ketumpatan udara diberikan sebagai 1.217 $kg/m^{3}$. Jawab soalan-soalan dibawah berdasarkan penjana angin tersebut.}
	

\listbegin

	\item Estimate the efficiency and the horizontal force exerted by the wind on the supporting mast of the wind turbine.   
	
	\translation{Anggarkan berapa kecekapan dan berapa daya mengufuk yang dikenakan oleh angin kepada tiang yang menyokong turbin angin itu.}

	\qmarks{5}

    \item Assuming that efficiency remains the same, evaluate the effect of doubling the wind speed on power generation and the force acting on the mast.   
	
	\translation{Dengan mengganggap kecekapan pada kadar yang sama, nilaikan kesan akibat peningkatan halaju angin sebanyak dua kali ganda keatas penjanaan kuasa dan daya yang bertindak keatas tiang sokongan.}

	\qmarks{5}

\listclose

		\begin{figure}[H] % H means, to put figure here after the code
		\centering
		\includegraphics[width=0.5\textwidth]{WindPowerGeneration}
		\caption{\rajah}
		\label{fig:WindPowerGeneration}
	    \end{figure}	


\listclose	% close 1st level question
