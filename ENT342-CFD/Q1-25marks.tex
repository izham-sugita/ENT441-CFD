%\setmainstyle
%\vskip -2em		% adjust vertical space skip accordingly 

%%%%%%%%%%%%%
%%% PART  %%%
%%%%%%%%%%%%%
%\textbf{Answer all questions} 
%
%\translationbf{Jawab semua soalan}
%\bigskip
%\newparten{Answer all questions}

%\newpartbm{Jawab semua soalan}


%%%%%%%%%%%%%%%%%%
%%% QUESTION 1 %%%
%%%%%%%%%%%%%%%%%%
%TODO Question1
%\bigskip 

\question{}[\label{q1label}]

\listbeginx	% start 1st level question
	\item \label{item:q1a} Sketch an appropriate figure, elaborate the difference between gage pressure and absolute pressure.
	
	\translation{Lukiskan gambarajah yang sesuai, terangkan perbezaan antara tekanan tolok dan tekanan mutlak.} 

\qmarks{5}	
	
	\item \label{item:q1b} Saturated water vapour at $150^{\circ}$C with enthalpy energy of h = 2745.9 kJ/kg flows in a pipe at 50 m/s at an elevation of z = 10m.
		
		\translation{Apakejadahnya translation melayu ni pun kena buat? awat English teruk sangat ka?}
		
		\listbegin 
		\item\label{item:q1b_1} Determine the total energy of vapour in kJ/kg relative to ground level.
		
		\translation{Tentukan jumlah tenaga wap dalam kJ/kg relatif kepada aras bumi.}
		
		\qmarks{3}	% define marks
		
		
		\item Based on Q\ref{q1label}\ref{item:q1b}\ref{item:q1b_1}, explain why both kinetic energy and potential energy can usually be neglected in most fluid flow analysis.
		
		\translation{Berdasarkan Q\ref{q1label}\ref{item:q1b}\ref{item:q1b_1}, terangkan kenapa kedua-dua tenaga kinetik dan tenaga keupayaan biasanya boleh diabaikan dalam kebanyakan analisis bendalir.}
		
		\qmarks{3}	% define marks		
		
		\listclose
		
		\item Pressure is often given in terms of a liquid column and is expressed as pressure head. Express the standard atmospheric pressure in terms of fluid column height for mercury (SG = 13.6), water (SG = 1.26) and glycerine (SG = 1.26). Based on your answer, explain which is the most suitable for manometer's application.
		
		\translation{Tekanan biasanya diberikan dari segi ketinggian bendalir dan dinyatakan sebagai turus tekanan. Nyatakan piawai tekanan atmosfera dalam bentuk ketinggian bendalir bagi raksa (SG = 13.6), air (SG = 1.0) dan gliserin (SG = 1.26)}
		
		\qmarks{6}

\clearpage

\item A hydraulic jack being used in a car repair workshop is shown in \textbf{Figure \ref{fig:hydraulicJack}}. The pistons have an area of $A_{1}$ = 0.8 $cm^{2}$ and $A_{2}$ = 0.04 $m^{2}$, respectively. Hydraulic oil with specific gravity (SG) of 0.87 is pressurized as the small piston on the left side is pushed up and down, raising the larger piston on the right side in order to jack up a car weighing 13000 N.

\translation{Satu bicu hidraulik yang digunakan dalam satu bengkel baik pulih kenderaan seperti yang ditunjukkan dalam Gambarajah \ref{fig:hydraulicJack}. Omboh-omboh mempunyai keluasan $A_{1}$ = 0.8 $cm^{2}$ dan $A_{2}$ = 0.04 $m^{2}$. Minyak hidraulik dengan graviti tentu bernilai 0.87 dipamkan masuk semasa omboh kecil di sebelah kiri ditekan ke atas dan ke bawah, menyebabkan kenaikan pada omboh besar dan menaikkan sebuah kenderaan seberat 13000 N.}

		\listbegin 
		\item Calculate the force $F_{1}$ in Newton that is required to hold the weight of the car when both pistons are at the same elevation (h = 0).
		
		\translation{Kirakan daya $F_{1}$ dalam Newton yang diperlukan untuk menampung berat kenderaan tersebut apabila kedua-dua omboh berada pada ketinggian yang sama (h=0).}
		
		\qmarks{3}	% define marks
		
		
		\item Evaluate what will happen to $F_{1}$ if the car is lifted to a certain elevation ( h > 0 ). Justify your answer.
		
		\translation{Nilaikan apa yang akan berlaku pada $F_{1}$ jika kenderaan tersebut akan dinaikkan ke aras yang tertentu ( h > 0 ). Justifikasikan jawapan anda.}
		
		\qmarks{5}	% define marks		
		
		\listclose
		
		\bigskip
		\begin{figure}[H] % H means, to put figure here after the code
		\centering
		\includegraphics[width=0.5\textwidth]{Q1_image}
		\caption{\rajah}
		\label{fig:Q1_image}
	\end{figure}

		
		
	%\listclose % close 2nd level		
\listclose % close 1st level question
