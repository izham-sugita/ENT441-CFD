%%%%%%%%%%%%%%%%%%
%%% QUESTION 6 %%%
%%%%%%%%%%%%%%%%%%
%TODO Question6
% \clearpage		% page break
\question{}\label{Q}
%\partQuestion{}

\listbeginx	% start 1st level question

	\item \label{a} There are many non-dimensional parameters in fluid mechanics. For example, the Reynolds number, which measure the ratio of inertial force compared to viscous force. The non-dimensional parameters offer a standardized way to interpret the physics of fluid mechanics. Answer the questions below regarding non-dimensional parameters.  
	
	\translation{Ada banyak parameter tidak berdimensi didalam mekanik bendalir. Sebagai contoh, nombor Reynolds, dimana nombor ini mengukur nisbah daya inertia berbanding daya kelikatan. Parameter tidak berdimensi memberikan cara untuk menginterpretasikan fizik mekanik bendalir secara berpiawai. Jawab soalan-soalan dibawah berkenaan parameter tidak berdimensi.}
		
		\listbegin

\item \label{i} The equation of motion for an object falling in a vacuum space is given as the equation below. Propose a suitable non-dimensional parameter for the equation and solve for both the dimensional parameter and non-dimensional parameter equations.

\translation {Persamaan gerakan untuk objek yang jatuh di dalam ruang vakum diberikan seperti persamaan di bawah. Cadangkan satu parameter tidak berdimensi untuk persamaan tersebut dan selesaikan untuk kedua-dua persamaan berdimensi dan tidak berdimensi.}

\begin{equation}
\nonumber
\frac{d^{2}z}{dt^{2}} = -g
\end{equation}
		
\qmarks{5}


	\item A practical non-dimensional parameter will give an insight of the ratio of significance for the physical phenomena that is being investigated. In this case, what is the interpretation of your non-dimensional parameter and its utility? Give the supporting arguments.  		

	\translation {Parameter tidak berdimensi yang praktikal akan memberikan gambaran mengenai nisbah kepentingan sesuatu fenomena fizikal yang dikaji. Dalam kes ini, apakah intepretasi parameter tidak berdimensi dan kegunaannya? Berikan hujah-hujah sokongan anda.}		

\qmarks{5}
		
		\listclose

	
	\clearpage 
%	\nextpage 
\finalpage
	
	\item \label{b} \textbf{Figure \ref{fig:nonDimensionalAerofoil}} shows an airfoil that is used in an experiment to predict its lift characteristic. The chord length of the airfoil $L_{c}$ is 1.0 m and the platform area $A$, which is an area viewed from the top of the airfoil when the attack angle $\alpha = 0$ is 10.0 $m^{2}$. To test the airfoil, a model of 1/10 in size is built and tested in a wind tunnel.  
	
	\translation{Rajah \ref{fig:nonDimensionalAerofoil} menunjukkan satu aerofoil yang digunakan dalam satu experimen untuk meramalkan karakter angkat. Panjang kod aerofoil itu $L_{c}$ adalah 1.0 m dan luas platform $A$, iaitu kawasan yang dilihat dari atas aerofoil itu ketika sudut serangan $\alpha = 0$ adalah 10.0 $m^{2}$. Untuk menguji aerofoil itu, satu model bersaiz 1/10 telah dibina dan diuji dalam terowong angin.}
	

\listbegin

	\item \label{b1} The suitable non-dimensional parameter for this experiment is the lift coefficient. Define the lift coefficient and give the physical interpretation of the lift coefficient. Prove that the lift coefficient is a function of Reynolds number, Mach number and attack angle $\alpha$.    
	
	\translation{Parameter tidak berdimensi yang sesuai untuk experimen ini adalah pekali daya angkatan. Berikan definisi pekali daya angkatan dan berikan interpretasi fizikalnya. Buktikan bahawa pekali daya angkatan adalah fungsi kepada nombor Reynolds, nombor Mach dan sudut serangan $\alpha$.}

	\qmarks{5}
	
	\item The airfoil is designed to cruise at a velocity of $V$=50 m/s close to the ground where the temperature is $25^{\circ}$C. At this temperature, the speed of sound is approximately 346 m/s. What is the required wind speed inside the wind tunnel for the airfoil design to achieve dynamic similarity? Analyze the problems in achieving dynamic similarity and propose a solution.
	
	\translation {Aerofoil ini direka untuk halaju pelayaran $V$=50 m/s pada berhampiran permukaan dimana suhu udara adalah $25^{\circ}$C. Pada suhu ini, halaju bunyi adalah lebih kurang 346 m/s. Apakah halaju angin yang diperlukan didalam terowong angin supaya rekabentuk aerofoil itu mencapai kesamaan dinamik? Analisakan masalah-masalah yang dihadapi untuk mencapai kesamaan dinamik ini dan cadangkan satu penyelesaian.}

\qmarks{5}

\listclose

		\begin{figure}[H] % H means, to put figure here after the code
		\centering
		\includegraphics[width=0.4\textwidth]{nonDimensionalAerofoil}
		\caption{\rajah}
		\label{fig:nonDimensionalAerofoil}
	    \end{figure}	


\listclose	% close 1st level question

\paperend
