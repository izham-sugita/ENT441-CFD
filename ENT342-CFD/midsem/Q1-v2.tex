%\setmainstyle
%\vskip -2em		% adjust vertical space skip accordingly 

%%%%%%%%%%%%%
%%% PART  %%%
%%%%%%%%%%%%%
%\textbf{Answer all questions} 
%
%\translationbf{Jawab semua soalan}
%\bigskip
%\newparten{Answer all questions}

%\newpartbm{Jawab semua soalan}


%%%%%%%%%%%%%%%%%%
%%% QUESTION 1 %%%
%%%%%%%%%%%%%%%%%%
%TODO Question1
%\bigskip 

\question{}[\label{q1}]

The governing equations for inviscid compressible flow is the Euler equations. The equations can be derived from three conservations principle which are the mass conservation, momentum conservation and energy conservation. The principles can be easily demonstrated for one-dimensional flow. Answer the questions below regarding the one-dimensional Euler equations.

\translation{Persamaan menakluk untuk aliran tidak likat boleh mampat adalah persamaan Euler. Persamaan ini boleh diterbitkan daripada tiga prinsip keabadian iaitu prinsip keabadian untuk jisim, momentum dan tenaga. Prinsip-prinsip ini dapat ditunjukkan dengan mudah untuk aliran satu dimensi. Jawab soalan-solan dibawah berkaitan dengan persamaan Euler satu dimensi.}

\listbeginx	% start 1st level question
	\item \label{item:q1a} Explain the three principle of conservations and their relations to the control volume analysis. Sketch a suitable control volume for your explanation.
	
	\translation{Jelaskan tiga prinsip keabadian itu dan hubungannya dalam analisa isipadu kawalan. Lakarkan rajah isipadu kawalan yang sesuai untuk penjelasan anda.} 

\qmarks{10}	
	
	\item \label{item:q1b} Using the information from Q\ref{q1}\ref{item:q1a}, prove that the one-dimensional Euler equations can be written in the form below. Here, $\rho$, $u$, $p$, $e$ are density, x-axis velocity, pressure and total energy, respectively.
		
		\translation{Dengan menggunakan maklumat daripada Q\ref{q1}\ref{item:q1a},buktikan bahawa persamaan Euler satu dimensi boleh ditulis dalam bentuk dibawah. Disini, $\rho$, $u$, $p$, $e$ adalah masing-masingnya ketumpatan, halaju pada paksi-x, tekanan dan tenaga keseluruhan.  }
	
	\begin{equation}
	\frac{\partial \textbf{Q} }{\partial t} + \frac{\partial \textbf{E}}{\partial x} = 0 \nonumber
	\end{equation}
	
	\begin{equation}
	\textbf{Q} = \begin{bmatrix}
	\rho \\ \rho u \\ e
	\end{bmatrix}    \nonumber
\end{equation}

\begin{equation}	
	\textbf{E} = \begin{bmatrix}
\rho u \\ \rho u^2 + p \\ (e+p)u
\end{bmatrix}	\nonumber
	\end{equation}
		
		
		\qmarks{10}

\item There are \textbf{four (4)} unknown variables in the equations from Q\ref{q1}\ref{item:q1b} but only \textbf{three (3)} equations, meaning that the system needs another equation to close it. The pressure $p$ and total energy $e$ are linked together by the state equation of ideal gas. Demonstrate the relation between the two quantities.

\translation{Terdapat empat (4) anu dalam kesemua persamaan daripada Q\ref{q1}\ref{item:q1b} tetapi cuma tiga (3) persamaan, bermakna sistem persamaan ini memerlukan satu lagi persamaan untuk dipenuhi. Tekanan $p$ dan tenaga keseluruhan $e$ adalah berkaitan dengan persamaan keadaan untuk gas unggul. Tunjukkan hubungan antara kedua-dua kuantiti itu.}		
		
		\qmarks{5}
\listclose % close 1st level question

