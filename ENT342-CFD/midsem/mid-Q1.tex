%\setmainstyle
%\vskip -2em		% adjust vertical space skip accordingly 

%%%%%%%%%%%%%
%%% PART  %%%
%%%%%%%%%%%%%
%\textbf{Answer all questions} 
%
%\translationbf{Jawab semua soalan}
%\bigskip
%\newparten{Answer all questions}

%\newpartbm{Jawab semua soalan}


%%%%%%%%%%%%%%%%%%
%%% QUESTION 1 %%%
%%%%%%%%%%%%%%%%%%
%TODO Question1
%\bigskip 

\question{}[\label{q1}]

The governing equation for fluid flow can be derived from the conservation laws. Using the concept of mass and momentum conservation, answer the questions below.

\translation{Persamaan menakluk untuk aliran bendalir boleh diterbitkan menggunakan hukum keabadian. Dengan menggunakan konsep keabadian jisim dan momentum, jawab soalan-soalan dibawah.}

\listbeginx	% start 1st level question
	\item \label{item:q1a} Derive the one-dimensional differential form of mass and momentum conservation for fluid flow using a suitable control volume.
	
	\translation{Terbitkan bentuk pembezaan satu dimensi untuk keabadian jisim dan momentum bagi aliran bendalir menggunakan isipadu kawalan yang sesuai.} 

\qmarks{10}	
	
	\item \label{item:q1b} Using the information from Q\ref{q1}\ref{item:q1a}, prove that the one-dimensional Euler equations can be written in the form below. Here, $\rho$, $u$, $p$, $e$ are density, x-axis velocity, pressure and total energy, respectively.
		
		\translation{Dengan menggunakan maklumat daripada Q\ref{q1}\ref{item:q1a},buktikan bahawa persamaan Euler satu dimensi boleh ditulis dalam bentuk dibawah. Disini, $\rho$, $u$, $p$, $e$ adalah masing-masingnya ketumpatan, halaju pada paksi-x, tekanan dan tenaga keseluruhan.  }
	
\qmarks{10}	


\listclose % close 1st level question

