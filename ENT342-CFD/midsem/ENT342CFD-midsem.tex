%% This document shows an example of writing exam paper in a single file.
%%
\documentclass[12pt]{article}
%\usepackage{finalxm}			% usually final exam. solid hours without minutes
\usepackage[minutes]{finalxm}	% show hours & minutes
%\usepackage[answers]{finalxm}	% answer scheme
%\usepackage{pdflscape}	% for landscape layout on specific page 

\usepackage{lipsum}

\usepackage{amsmath}
\usepackage{amssymb}% for \mathbb

%\usepackage{mathptmx}
%\usepackage{mathpazo}


% Declare graphics path 
 \graphicspath{{./Figures/}}	% a subfolder named figs

%%%%%%%%%%%%%%%%%%%%%%%%%%%
%%% PERSONALIZED STYLES %%%
%%%%%%%%%%%%%%%%%%%%%%%%%%%

%% WITH CAPTION %%
%% Options availabe for caption. put these before begin{figure/table}
%% font, labelfont, textfont || use either bf, normalfont
%% labelsep || use either newline, colon
%\captionsetup[figure]{labelsep=newline, font=normalfont} 	% all bold
%\captionsetup[table]{labelsep=newline, font=normalfont}	% all bold
%\captionsetup[figure]{labelsep=colon} 	% all bold
%\captionsetup[table]{labelsep=colon}	% all bold
%\captionsetup[figure]{labelsep=colon, textfont=normalfont} 	% bold Figure
%\captionsetup[table]{labelsep=colon, textfont=normalfont} 	% bold Table
%\captionsetup[figure]{labelsep=colon, font=normalfont} 	% all no bold 
%\captionsetup[table]{labelsep=colon, font=normalfont} 	% all no bold 

%%%%%%%%%%%%%%%%%%%%%%%%
%%% SETUP TITLE PAGE %%%
%%%%%%%%%%%%%%%%%%%%%%%%
\mtefe{Pertengahan} 			% Akhir, Pertengahan 
\semester{Kedua}		% Pertama, Kedua
\sidang{2018/2019}		
\examMonthYear{Oktober 2019}
\courseCode{ENT342}
\courseNameEn{Computational Fluid Dynamics}
\courseNameBM{Pengiraan Dinamik Bendalir}
\pagesbm{TIGA PULUH SATU}	% used if the number of pages exceeds 30
\durationhr{1 Jam 30 minit}		% duration of exam
\makeatletter 
%\if@minutes
%	\durationmin{30 Minit}	% MUST be enabled using \usepackage[minutes]{finalxm}
%\fi 		
\makeatother		

%test
\DeclareMathAlphabet{\mathpzc}{OT1}{pzc}{m}{it}

\begin{document}
\makecover	% make cover/title page

\instructionen{
This question paper has \textbf{TWO (2)} questions. Answer \textbf{ALL} of the questions. Each question contributes 25 marks.}
\instructionbm{Kertas soalan ini mengandungi \textbf{DUA (2)} soalan. Jawab \textbf{semua} soalan. Markah bagi setiap soalan adalah 25 markah.}

%Note: Some tables and equations are given in the Appendix
%[Nota : Beberapa jadual dan persamaan diberikan dalam Lampiran]

% Delete from here if require no extra notes 
%\vskip 3em
%\instructionen{
%	Note: Tables and equations are given in the Appendix.%
%}
%\instructionbm{%
%	Nota: Jadual dan persamaan diberi dalam Lampiran.%
%}
% Delete until here if require no extra notes 

\makecoverend	% end cover/title page

%%%%%%%%%%%%%%%%%%%%%%%%%%%%
%%% MAIN BODY START HERE %%%
%%%%%%%%%%%%%%%%%%%%%%%%%%%%
\setmainstyle
\vskip -2em		% adjust vertical space skip accordingly 

%%%%%%%%%%%%%
%%% PART  %%%
%%%%%%%%%%%%%
%\textbf{Answer all questions} 
%
%\translationbf{Jawab semua soalan}
%\bigskip
% \newparten{Answer all questions}

%\newpartbm{Jawab semua soalan}


%%%%%%%%%%%%%%%%%%
%%% QUESTION 1 %%%
%%%%%%%%%%%%%%%%%%
%TODO Question1
 \bigskip 
%\partQuestion{[CO1, PO1]}

%\setmainstyle
%\vskip -2em		% adjust vertical space skip accordingly 

%%%%%%%%%%%%%
%%% PART  %%%
%%%%%%%%%%%%%
%\textbf{Answer all questions} 
%
%\translationbf{Jawab semua soalan}
%\bigskip
%\newparten{Answer all questions}

%\newpartbm{Jawab semua soalan}


%%%%%%%%%%%%%%%%%%
%%% QUESTION 1 %%%
%%%%%%%%%%%%%%%%%%
%TODO Question1
%\bigskip 

\question{}[\label{q1}]

The governing equation for fluid flow can be derived from the conservation laws. Using the concept of mass and momentum conservation, answer the questions below.

\translation{Persamaan menakluk untuk aliran bendalir boleh diterbitkan menggunakan hukum keabadian. Dengan menggunakan konsep keabadian jisim dan momentum, jawab soalan-soalan dibawah.}

\listbeginx	% start 1st level question
	\item \label{item:q1a} Derive the one-dimensional differential form of mass and momentum conservation for fluid flow using a suitable control volume.
	
	\translation{Terbitkan bentuk pembezaan satu dimensi untuk keabadian jisim dan momentum bagi aliran bendalir menggunakan isipadu kawalan yang sesuai.} 

\qmarks{10}	
	
	\item \label{item:q1b} Using the information from Q\ref{q1}\ref{item:q1a}, prove that the one-dimensional Euler equations can be written in the form below. Here, $\rho$, $u$, $p$, $e$ are density, x-axis velocity, pressure and total energy, respectively.
		
		\translation{Dengan menggunakan maklumat daripada Q\ref{q1}\ref{item:q1a},buktikan bahawa persamaan Euler satu dimensi boleh ditulis dalam bentuk dibawah. Disini, $\rho$, $u$, $p$, $e$ adalah masing-masingnya ketumpatan, halaju pada paksi-x, tekanan dan tenaga keseluruhan.  }
	
\qmarks{10}	


\listclose % close 1st level question


\clearpage
%\nextpage


%\clearpage		% page break
\question{}\label{mid-q2}
%\partQuestion{}

The one-dimensional linear advection equation is given as below.  
	
	\translation{Persamaan adveksi linear satu dimensi diberikan seperti dibawah.}
	
	\begin{equation}
	\frac{\partial u}{\partial t} + c \frac{\partial u}{\partial x} = 0, \quad c > 0 \nonumber
	\end{equation}		

\listbeginx	% start 1st level question			

    \item \label{mid-q2a} The advection equation can be discretized using forward-time centered-space (FTCS) scheme. Derive the error amplification factor of the scheme to prove its stability.	
	\translation{Persamaan adveksi linear boleh didiskritkan menggunakan skim masa-kehadapan ruang-pertengahan (FTCS). Terbitkan faktor amplikasi ralat skim itu untuk membuktikan kestabilannya.}


\qmarks{10}


		
	\item The advection equation can also be discretized using the backward-time integration instead of forward-time. Derive the discrete equation and prove its stability.   
	
	\translation{Persamaan adveksi linear juga boleh didiskritkan menggunakan integrasi masa-mundur dan bukannya masa-kehadapan. Terbitkan persamaan diskrit tersebut dan buktikan kestabilannya. }

	\qmarks{10}
		
\listclose	% close 1st level question



%TODO PART B
%\clearpage
%\newpartenx{Answer TWO (2) questions ONLY}

%\newpartbm{Jawab DUA (2) soalan SAHAJA}
%\bigskip 

%\question{}
%\partQuestion{}

%%%%%%%%%%%%%%%%%%%
%%% QUESTION 5 %%%
%%%%%%%%%%%%%%%%%%
%TODO Question5
% \clearpage		% page break
\question{} \label{q5}
%\partQuestion{}

	The time-dependent one-dimensional heat equation is a second-order partial differential equation. The second-order term infers that the solution propagation will be restricted to the order of $O$ $(1/\Delta h^{2})$, where $\Delta h$ is the mesh spacing. This put a severe stability restriction to explicit numerical scheme when solving this equation. Answer the questions referring to the time-dependent one-dimensional heat equation below.
	
	\translation{Persamaan haba satu-dimensi bersandar masa merupakan persamaan kebezaan separa darjah kedua. Terma darjah kedua bererti propagasi solusi disekat pada darjah $\mathcal{O}(1/\Delta h^{2})$, dimana $\Delta h$ adalah jarak antara grid. Ini menyebabkan penyekatan syarat kestabilan yang teruk bagi kaedah berangka tersurat ketika menyelesaikan persamaan ini. Jawab soalan-soalan berkaitan dengan persamaan haba satu-dimensi bersandar masa dibawah.}

\begin{equation}
\frac{\partial T}{\partial t} = \kappa \left( \frac{\partial^2 T}{\partial x^2} \right) \nonumber
\end{equation}		
		
		\listbeginx

\item \label{q5a} Construct the discretized equation for the heat equation using \textit{forward} Euler time integration and centered difference scheme for the spatial term. Analyze the discrete equation to derive the numerical stability condition.

\translation {Bina persamaan diskrit untuk persamaan haba itu menggunakan kaedah integrasi masa Euler kehadapan dan kaedah pembezaan pertengahan untuk terma ruang. Analisakan persamaan diskrit itu untuk menerbitkan syarat kestabilan berangka.}
		
\qmarks{10}

\item \label{q5b} From the discrete equation obtained in Q\ref{q5}\ref{q5a} the time integration scheme can be modified by setting the time step forward or backward. Answer the questions below regarding backward time step method.     

\translation{Daripada persamaan diskrit yang diperolehi oleh Q\ref{q5}\ref{q5a} kaedah integrasi masa boleh diubahsuai dengan menetapkan langkah masa kehadapan atau mundur. Jawab soalan-soalan dibawah berkenaan kaedah mundur langkah masa.}

\listbegin
\item Apply the \textit{backward} Euler time integration to the discrete equation obtained in Q\ref{q5}\ref{q5a} and rearrange the equation.

\translation{Applikasikan kaedah integrasi masa mundur Euler keatas persamaan diskrit yang diperolehi daripada Q\ref{q5}\ref{q5a} dan susun semula persamaan itu.}

\qmarks{5}

\item Analyze the stability condition for the backward Euler time integration.

\translation{Analisakan syarat kestabilan kaedah integrasi masa mundur Euler.}

\qmarks{5}

\listclose

	
	\item The discrete equation from Q\ref{q5}\ref{q5b} contains more unknown at the next time step than at the previous time step. This condition necessitates the construction of simultaneous linear equations. Construct the matrix-vector form for the simultaneous linear equations.  
	
	\translation{Persamaan diskrit daripada Q\ref{q5}\ref{q5b} mengandungi lebih banyak terma di masa seterusnya daripada masa sebelumnya. Keadaan ini memerlukan pembinaan persamaan linear serentak. Bina persamaan linear serentak itu dalam bentuk matriks-vektor.}

	\qmarks{5}



\listclose	% close 1st level question



\paperend 

%\clearpage

%\paperend %<- included in Q6.tex
%\clearpage
\newpage
\textbf{\underline{Appendices}}
\begin{figure}[H] % H means, to put figure here after the code
\centering
\includegraphics[width=\textwidth]{appendix1}
\label{fig:appendix1}
\end{figure}	

\begin{figure}[H] % H means, to put figure here after the code
\centering
\includegraphics[width=\textwidth]{appendix2}
\label{fig:appendix2}
\end{figure}



\end{document}