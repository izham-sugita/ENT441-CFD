%%%%%%%%%%%%%%%%%%
%%% QUESTION 2 %%%
%%%%%%%%%%%%%%%%%%
%TODO Question2
\clearpage		% page break
\question{}
%\partQuestion{}

\listbeginx	% start 1st level question
	\item Water in a tank is pressurized by air and the pressure is measured by a multi-fluid manometer as shown in \textbf{Figure \ref{fig:pressurizedTank}}. Analyze the gage pressure of air in the tank if $h_{1}$ = 0.4 m, $h_{2}$ = 0.6 m, and $h_{3}$ = 0.8 m. The densities of water, oil and mercury are given as 1000 kg/$m^{3}$, 850 kg/$m^{3}$ and 13,600 kg/$m^{3}$, respectively.  
	
	\translation{Air dalam satu tangki diberi tekanan udara dan tekanan tersebut diukur  dengan manometer berbilang bendalir seperti yang ditunjukkan dalam \textbf{Gambarajah \ref{fig:iceCube}}. Nilaikan tekanan tolok bagi air sekiranya $h_{1}$ = 0.4 m, $h_{2}$ = 0.6 m, and $h_{3}$ = 0.8 m. Ketumpatan bagi air, minyak dan raksa adalah 1000 kg/$m^{3}$, 850 kg/$m^{3}$ dan 13,600 kg/$m^{3}$.  }
	
		
		\qmarks{5}

        \begin{figure}[H] % H means, to put figure here after the code
		\centering
		\includegraphics[width=0.4\textwidth]{pressurizedTank2}
		\caption{\rajah}
		\label{fig:pressurizedTank}
	    \end{figure}
	
%	\nextpage 
%	\clearpage 
	
	\item Consider a large cubic ice block floating in seawater as shown in \textbf{Figure \ref{fig:iceCube}}. The specific gravity of ice and seawater are 0.92 and 1.025, respectively. If a 25 cm high portion of the ice block extends above the surface water, determine the height of the ice block below the surface.   
	
	\translation{Anggapkan satu kiub ais besar terapung di laut seperti yang ditunjukkan dalam Gambarajah \ref{fig:iceCube}. Graviti tentu ais dan air laut masing-masing adalah 0.92 dan 1.025. Jika sebanyak 25 cm dari ais kiub tersebut terkeluar melebihi permukaan air laut, tentukan ketinggian kiub ais tersebut yang berada di bawah permukaan air laut.}
	

\qmarks{5}	
	
	\begin{figure}[H] % H means, to put figure here after the code
		\centering
		\includegraphics[width=0.5\textwidth]{iceCube2}
		\caption{\rajah}
		\label{fig:iceCube}
	    \end{figure}

	

\item An 80 cm high aquarium with cross-section 2 m $\times$ 0.6 m is partially filled with water as shown in \textbf{Figure \ref{fig:aquarium}} The aquarium is to be transported on the back of a truck. The truck accelerates from 0 to 90 km/h in 10s. Assume that acceleration remains constant and horizontal.  
	
	\translation{Satu akuarium setinggi 80 cm yang mempunyai keratan rentas 2 m $\times$ 0.6 m diisi air separa penuh seperti dalam Gambarajah \ref{fig:aquarium}. Akuarium tersebut diletakkan di atas lori. Lori tersebut memecut dari 0 sehingga 90 km/h dalam masa 10s. Anggapkan pecutan adalah kekal dan dalam keadaan mengufuk. }

\listbegin

\item If it is desired that no water spills during acceleration, analyze the allowable initial water height in the tank.
\translation{Sekiranya tumpahan air tidak boleh berlaku semasa pecutan, analisakan ketinggian mula air dalam tangki yang dibenarkan.}

\qmarks{5}

\item To minimize risk of water spill from aquarium, would you propose the tank to be aligned with the long or short side parallel to the direction of motion?
\translation{Untuk meminimumkan risiko air terpercik keluar dari akuarium, adakan anda akan mencadangkan bahawa tangki tersebut akan dijajar samada dengan bahagian panjang atau pendek yang selari dengan arah gerakan? }

\qmarks{5}

\listclose


	\begin{figure}[H] % H means, to put figure here after the code
		\centering
		\includegraphics[width=0.5\textwidth]{aquarium2}
		\caption{\rajah}
		\label{fig:aquarium}
	    \end{figure}


\listclose	% close 1st level question
