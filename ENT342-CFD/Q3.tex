%%%%%%%%%%%%%%%%%%
%%% QUESTION 3 %%%
%%%%%%%%%%%%%%%%%%
%TODO Question3
\clearpage		% page break
\question{}\label{Bernoulli}
%\partQuestion{}

\listbeginx	% start 1st level question

	\item \label{Bernoulli_1}The Bernoulli equation is a very useful tool in fluid mechanics. It is a simplified version of the Navier-Stokes equations for certain conditions. The simplification is what made the equation useful as a design tool in fluid mechanics.  
	
	\translation{Persamaan Bernoulli adalah satu persamaan yang sangat berguna dalam mekanik bendalir. Persamaan ini merupakan satu persamaan yang dimudahkan daripada persamaan Navier-Stokes dalam keadaan-keadaan tertentu. Permudahan inilah yang membuatkan persamaan ini sangat berguna dalam rekabentuk melibatkan mekanik bendalir.}
	
		
		\listbegin

	\item In your own words, describe the Bernoulli equation.   
	
	\translation{Dalam perkataan anda sendiri, terangkan apakah persamaan Bernoulli.}

	\qmarks{2}

    \item \label{Bernoulli_2}The Bernoulli equation is derived from basic mechanics principles together with a few assumptions for simplification. Explain and describe the methodology.   
	
	\translation{Persamaan Bernoulli ditakkulkan daripada prinsip-prinsip asas mekanik dengan menggunakan beberapa anggapan untuk memudahkan persamaan. Jelaskan dan terangkan metodologinya.}

	\qmarks{3}
		
	\item Demonstrate the step-by-step of deriving the Bernoulli equation by applying the methodology stated in Q\ref{Bernoulli}\ref{Bernoulli_1}\ref{Bernoulli_2}. Use the steady flow assumption.   
	
	\translation{Tunjukkan satu per satu langkah-langkah untuk mendapatkan persamaan Bernoulli menggunakan metodologi daripada Q\ref{Bernoulli}\ref{Bernoulli_1}\ref{Bernoulli_2}. Gunakan anggapan aliran mantap.}

	\qmarks{3}

    \item Generally, fluid flows can be categorized into incompressible and compressible flows. Using the derived Bernoulli equation, differentiate the equation for incompressible and compressible flow.   
	
	\translation{Umumnya, aliran bendalir boleh dikategorikan kepada aliran tidak mampat dan aliran boleh mampat. Dengan menggunakan persamaan Bernoulli yang diperolehi, bezakan persamaan itu antara aliran tidak mampat dan aliran boleh mampat.}

	\qmarks{2}
		
		\listclose
	
	\clearpage 
%	\nextpage 

	
	\item Torricelli's law describes the flow velocity of fluid discharging from a reservoir. The law is a canonical example of the Bernoulli equation's application. \textbf{Figure \ref{fig:Bernoulli-Torricelli}} shows a typical water reservoir, with $D_{tank}$ as the diameter of the reservoir, $D_{jet}$ as the diameter of discharging water jet and $h_{0}$ as the maximum water level. The water level reference axis is given as $h$. For this question, your solution should be in the algebraic form only.   
	
	\translation{Hukum Torricelli menyatakan bagaimana halaju bendalir yang mengalir keluar dari takungan. Hukum ini adalah satu contoh bersejarah mengenai aplikasi persamaan Bernoulli. Gambarajah \ref{fig:Bernoulli-Torricelli} menunjukkan contoh kebiasaan takungan air, dimana $D_{tank}$ adalah diameter tangki, $D_{jet}$ adalah diameter jet air yang mengalir dan $h_{0}$ adalah tinggi maksima takungan air. Paksi rujukan ketinggian air diberikan sebagai $h$. Untuk soalan ini, solusi anda hanya akan berbentuk algebra.}
	

\listbegin

	\item Using Bernoulli equation, construct the exiting water's velocity from the tank as a function of height $h$. Derive the special case of Torricelli's law by using the assumption $D_{tank}$ $\gg$ $D_{jet}$.   
	
	\translation{Menggunakan persamaan Bernoulli, binakan persamman untuk halaju aliran air yang keluar dari tangki sebagai fungsi kepada ketinggian $h$. Takkulkan kes istimewa untuk hukum Torricelli dengan menggunakan andaian $D_{tank}$ $\gg$ $D_{jet}$.}

	\qmarks{5}

    \item Estimate the time required for the water level of the tank to reach the height of $h_{2}$.   
	
	\translation{Anggarkan masa yang diperlukan untuk ketinggian air mencapai $h_{2}$.}

	\qmarks{5}

\listclose

		\begin{figure}[H] % H means, to put figure here after the code
		\centering
		\includegraphics[width=0.5\textwidth]{Bernoulli-Torricelli}
		\caption{\rajah}
		\label{fig:Bernoulli-Torricelli}
	    \end{figure}	


\listclose	% close 1st level question
